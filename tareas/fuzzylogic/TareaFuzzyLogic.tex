
\documentclass[12pt]{article}
\usepackage[latin 1]{input enc} %Biblioteca de espa�ol
\usepackage{latexsym}
\usepackage[spanish]{babel} %Biblioteca de espa�ol
\usepackage[dvips]{epsfig} %Biblioteca para insertar figuras
\usepackage{listings}
\usepackage{color}
\usepackage[dvips]{epsfig} %Biblioteca para insertar figuras
\usepackage{url}
\usepackage{amssymb}


\lstset{language=Matlab, backgroundcolor=\color{white}, basicstyle=\small, commentstyle=\color{blue},frame=single,title=\lstname}

 %A continuaci�n se presenta algunos c�digos para cambiar el formato de la hoja.
 \topmargin  -1.8 cm
 \setlength{\paperheight}{27.7cm}
 \setlength{\textheight}{23.7cm}
 \setlength{\paperwidth}{21cm}
 \oddsidemargin    0 cm \evensidemargin   0cm
 \setlength{\textwidth}{17cm}
 \newlength{\defbaselineskip}
 \setlength{\defbaselineskip}{\baselineskip}
 \newcommand{\setlinespacing}[1]%
           {\setlength{\baselineskip}{#1 \defbaselineskip}}
 \newcommand{\doublespacing}{\setlength{\baselineskip}%
                           {1.5 \defbaselineskip}}
 \newcommand{\singlespacing}{\setlength{\baselineskip}{\defbaselineskip}}

 

\title{\textbf{Tarea Fuzzy Logic} \\Inferencia de Conocimiento en Sistemas de Informaci�n\\ Profesor: Dr.(c)
H�ctor Allende}

\author{Paola Arce\\ \textit{paola.arce@usm.cl}\\ Departamento de
Inform�tica \\ Universidad T�cnica Federico Santa Mar�a}



\begin{document}
\maketitle
\section{Pregunta 1}
\subsection{Operadores t-norma} \textit{Verificar que los operadores PAND (t-norm probabilistic) y LAND (t-norm de Lukasiewics) cumplen las propiedades de t-norma.} \\
\newline
Un operador t-norm debe satisfacer las siguientes condiciones:

\begin{enumerate}
	\item{{\color{red}Condiciones de borde}}
	 	\begin{enumerate}
 			\item $T(0,0) = 0$
				\begin{enumerate}
					\item $PAND(0,0)=0*0=0 $   
					 $\therefore$  se cumple
					\item $LAND(0,0)=max(0+0-1,0)=0$
					$\therefore$  se cumple
				\end{enumerate}
			\item $T(a,1)=T(1,a)=a$
				\begin{enumerate}
					\item $PAND(a,1)=a*1=a \quad \wedge \quad  PAND(1,a)=1*a=a$   
					 $\therefore$  se cumple
					\item $LAND(a,1)=max(a+1-1,0)=a \hspace{2mm} \wedge \hspace{2mm}LAND(1,a)=max(1+a-1,0)=a$
					$\therefore$  se cumple
				\end{enumerate}
		\end{enumerate}
	\item{{\color{red}Monotonia}} $T(a,b) \leq T(c,d) \quad $ if $ \quad a \leq c \wedge b \leq d$
		\begin{enumerate}
					\item $PAND(a,b)=ab \quad \wedge \quad PAND(c,d)=cd$
					
					Dado que $ a \leq c \wedge b \leq d$ se tiene que $ab \leq cd$
					 $$\therefore PAND(a,b)=ab \leq PAND(c,d)=cd \quad \mbox{se cumple}$$
					\item $LAND(a,b)=max(a+b-1,0) \quad \wedge \quad LAND(c,d)=max(c+d-1,0)$
					
					Dado que $ a \leq c \wedge b \leq d$ se tiene que $a+b \leq c+d$ y $a+b -1\leq c+d-1$
					
					De aqui se tiene que $max(a+b-1,0) \leq max(c+d-1,0)$
					
					$\therefore LAND(a,b) \leq LAND(c,d)$  se cumple
				\end{enumerate}
	\item{{\color{red}Conmutatividad}} $T(a,b)=T(b,a)$
		\begin{enumerate}
					\item $PAND(a,b)=ab \quad \wedge \quad  PAND(b,a)=ba=ab$   
					 $\therefore$  se cumple
					\item $LAND(a,b)=max(a+b-1,0)=a \hspace{2mm} \wedge \hspace{2mm}LAND(b,a)=max(b+a-1,0)=max(a+b-1,0)$
					$\therefore$  se cumple
				\end{enumerate}
	\item{{\color{red}Asociatividad}} $T(a,T(b,c))=T(T(a,b),c)$
		\begin{enumerate}
			\item $PAND(a, PAND(b,c))=a*PAND(b,c)=abc \quad \wedge \quad PAND(PAND(a,b),c)=PAND(a,b)*c=abc$  $\therefore$  se cumple
			\item $LAND(a,LAND(b,c))=max(a+max(b+c-1,0)-1,0)$
			
			$=\left\{
				\begin{array}{ll}
					max(a+b+c-2,0)=a+b+c-2  & \mbox{if } a+b \geq 1 \wedge b+c \geq 1\\
					max(a+b+c-2,0)=0 & \mbox{if } a+b \leq 1 \wedge b+c \geq 1\\
					max(a-1,0)=0 & \mbox{if } a+b \geq 1 \wedge b+c \leq 1\\
					max(a-1,0)=0 & \mbox{if } a+b \leq 1 \wedge b+c \leq 1\\
				\end{array}
			\right.$
			
			$LAND(LAND(a,b),c)=max(max(a+b-1,0)+c-1,0)$
			
			$=\left\{
				\begin{array}{ll}
					max(a+b+c-2,0)=a+b+c-2  & \mbox{if } a+b \geq 1 \wedge b+c \geq 1\\
					max(c-1,0)=0 & \mbox{if } a+b \leq 1 \wedge b+c \geq 1\\
					max(a+b+c-2,0)=0 & \mbox{if } a+b \geq 1 \wedge b+c \leq 1\\
					max(c-1,0)=0 & \mbox{if } a+b \leq 1 \wedge b+c \leq 1\\
				\end{array}
			\right.$
			
			$\therefore$ se cumple. Ambas funciones son iguales en cada caso.			
		\end{enumerate}
\end{enumerate}



\subsection{Operadores s-norma} Verificar que los operadores POR (s-norm probabilistic) y LOR (s-norm de Lukasiewics) cumplen las propiedades de s-norma.

Un operador s-norm debe satisfacer las siguientes condiciones:
\begin{enumerate}
	\item{{\color{red}Condiciones de borde}}
	 	\begin{enumerate}
 			\item $S(1,1) = 1$
				\begin{enumerate}
					\item $POR(1,1)=1+1-1=1 $   
					 $\therefore$  se cumple
					\item $LOR(0,0)=min(1+1,1)=1$
					$\therefore$  se cumple
				\end{enumerate}
			\item $S(a,0)=S(0,a)=a$
				\begin{enumerate}
					\item $POR(a,0)=a+0-0=a \quad \wedge \quad  POR(0,a)=0+a-0=a$   
					 $\therefore$  se cumple
					\item $LOR(a,0)=min(a+0,1)=a \hspace{2mm} \wedge \hspace{2mm}LOR(0,a)=min(0+a,1)=a$
					$\therefore$  se cumple
				\end{enumerate}
		\end{enumerate}
	\item{{\color{red}Monotonia}} $S(a,b) \leq S(c,d) \quad $ if $ \quad a \leq c \wedge b \leq d$
		\begin{enumerate}
					\item $POR(a,b)=a+b-ab \quad \wedge \quad POR(c,d)=c+d-cd$
					p.d $a+b-ab \leq c+d -cd$ lo que equivale a demostrar que
					
					$$c+d-cd -(a+b-ab)\geq 0 $$
					$$(c-a) + d(1-c) -b(1-a) \geq (c-a) + b(1-c) -b(1-a) = c-a+b-bc-b+ab=c-a-bc+ab$$
					$$(c-a) + d(1-c) -b(1-a) \geq (c-a)(1-b)$$
					Los t�rminos $(c-a)$ y $(1-b)$ son $\geq 0$ por definici�n
					
					$\therefore \quad (c-a) + d(1-c) -b(1-a) \geq 0$ 
					
					\item $LOR(a,b)=min(a+b,1) \quad \wedge \quad LOR(c,d)=min(c+d,1)$
					
					Dado que $ a+b \leq c+d$ se tiene que:					
					$min(a+b,1) \leq min(c+d,1)$
					
					$\therefore LOR(a,b) \leq LOR(c,d)$  se cumple
				\end{enumerate}
	\item{{\color{red}Conmutatividad}} $S(a,b)=S(b,a)$
		\begin{enumerate}
					\item $POR(a,b)=a+b-ab \quad \wedge \quad  POR(b,a)=b+a-ab=a+b-ab$   
					 $\therefore$  se cumple
					\item $LOR(a,b)=min(a+b,1) \hspace{2mm} \wedge \hspace{2mm}LOR(b,a)=min(b+a,1)=min(a+b,1)$
					$\therefore$  se cumple
				\end{enumerate}
	\item{{\color{red}Asociatividad}} $S(a,S(b,c))=S(S(a,b),c)$
		\begin{enumerate}
			\item $POR(a, POR(b,c))=a+POR(b,c)-a*POR(b,c)=a+b+c-bc-ab-ac+abc 
			\quad \wedge \quad POR(POR(a,b),c)=POR(a,b)+c-POR(a,b)*c=a+b+c-bc-ab-ac+abc $  $\therefore$  se cumple
			\item $LOR(a,LOR(b,c))=min(a+min(b+c,1),1)$
			
			$=\left\{
				\begin{array}{ll}
					min(a+1,1)=1  & \mbox{if } a+b \geq 1 \wedge b+c \geq 1\\
					min(a+1,1)=1 & \mbox{if } a+b \leq 1 \wedge b+c \geq 1\\
					min(a+b+c,1)=1 & \mbox{if } a+b \geq 1 \wedge b+c \leq 1\\
					min(a+b+c,1)=\left\{  
					\begin{array}{ll}
					a+b+c  & \mbox{if } a+b+c \leq 1\\
					1 & \mbox{if } a+b+c \geq 1
					\end{array}\right.
				 & \mbox{if } a+b \leq 1 \wedge b+c \leq 1
				\end{array}
			\right.$
			
			$LOR(LOR(a,b),c)=min(min(a+b,1)+c,1)$
			
		$=\left\{
				\begin{array}{ll}
					min(1+c,1)=1  & \mbox{if } a+b \geq 1 \wedge b+c \geq 1\\
					min(a+b+c,1)=1 & \mbox{if } a+b \leq 1 \wedge b+c \geq 1\\
					min(1+c,1)=1 & \mbox{if } a+b \geq 1 \wedge b+c \leq 1\\
					min(a+b+c,1)=\left\{  
					\begin{array}{ll}
					a+b+c  & \mbox{if } a+b+c \leq 1\\
					1 & \mbox{if } a+b+c \geq 1
					\end{array}\right.
				 & \mbox{if } a+b \leq 1 \wedge b+c \leq 1
				\end{array}
			\right.$
			
			$\therefore$ se cumple. Ambas funciones son iguales en cada caso.			
		\end{enumerate}
\end{enumerate}



\section{Pregunta 2}
Verificar si las siguientes propiedades se cumplen para Fuzzy sets. Use las t-norm y s-norm de problema
1 en caso de no ser v�lida la propiedad mostrar con un contra ejemplo.

\begin{itemize}
\item {\color{red}$\overline{A \cup B} = \overline{A} \cap \overline{B}$}
\begin{enumerate}
\item{Usando PAND y POR}\newline
$\overline{A \cup B} = \overline{POR(A,B)}=1- A-B-AB$ \newline
$ \overline{A} \cap \overline{B} = PAND(1-A,1-B)=(1-A)*(1-B)=1-A-B-AB$ \newline
$\therefore$ se cumple
\item{Usando LAND y LOR} \newline
$\overline{A \cup B} = \overline{LOR(A,B)}=1- min(A+B,1) = \left\{  
					\begin{array}{ll}
					1-1=0 & \mbox{if } A+B \geq 1\\
					1-A-B & \mbox{if } A+B < 1
					\end{array}\right.
					$ \newline
$ \overline{A} \cap \overline{B} = LAND(1-A,1-B)=max(1-A+1-B-1,0)=$
                                              \newline
                                              $max(1-A-B,0)=\left\{  
					\begin{array}{ll}
					0 & \mbox{if } A+B \geq 1\\
					1-A-B & \mbox{if } A+B < 1
					\end{array}\right.
					$ \newline

$\therefore$ se cumple
\end{enumerate}



\item  {\color{red}$A \cup (B \cap C)=(A \cup B) \cap (A \cup C)$}
\begin{enumerate}
\item{Usando PAND y POR}\newline
$POR(A,PAND(B,C))=A+PAND(B,C)-A*PAND(B,C)=A+BC-ABC$ \newline
$POR(A,B)*POR(A,C)=(A+B-AB)*(A+C-AC)$\newline
$\therefore$ no son iguales

Un contra ejemplo se obtiene d�ndole valores a las funciones de pertenencia. Sea $A=0.1$, $B=0.8$ y $C=0.3$. Se tiene que $A+BC-ABC=0.316$ y que $(A+B-AB)*(A+C-AC)=0.3034$

\item{Usando LAND y LOR}\newline
$POR(A,PAND(B,C))=min(A+max(B+C-1,0),1)$\newline
$POR(A,B)*POR(A,C)=max(min(A+B,1)+min(A+C,1)-1,0)$\newline
Las funciones no son iguales, basta tomar el caso donde $B+C <1 \wedge A+B >1 \wedge A+C > 1$ se tendr�a que:
\newline
$POR(A,PAND(B,C))=min(A,1)=A$
$POR(A,B)*POR(A,C)=max(1+1-1,0)=1$
\newline
$\therefore$ no son iguales
\end{enumerate}

\item  {\color{red}$A \cup \overline{A}=X$}

\begin{itemize}
\item{Usando PAND y POR}\newline
$POR(A, 1-A)=A+(1-A)-A(1-A)=1-A+A^2 \neq 1$
$\therefore$ no se cumple

\item{Usando LAND y LOR}\newline
$LOR(A,1-A)=min(A+1-A,1)=1$
$\therefore$ se cumple
\end{itemize}

\item  {\color{red}$A \cup \overline{A}= \emptyset$}

\begin{itemize}
\item{Usando PAND y POR}\newline
$PAND(A, 1-A)=A*(1-A)= A- A^2 \neq \emptyset$
$\therefore$ no se cumple

\item{Usando LAND y LOR} \newline
$LAND(A,1-A)=max(A+1-A-1-1,0)=0$
$\therefore$ se cumple
\end{itemize}


\end{itemize}

\section{Pregunta 3}
Sean R y S relaciones Fuzzy de X en Y. Verificar que las relaciones Fuzzy cumplen las siguientes propiedades. Use las t-norm y s-norm de Mamdani.

\begin{enumerate}
\item {\color{red}$(R \cup S)^{-1} = R^{-1} \cup S^{-1}$}
\newline
Se define la uni�n como sigue:

$$\mu_{R \cup S}(x,y)=\mu_R(x,y) \vee \mu_S(x,y)$$
$$\mu_{(R \cup S)^{-1}}(y,x)=\mu_{R \cup S}(x,y)$$


Por otra parte $R^{-1} \cup S^{-1}$ se define:

$$\mu_{R_{-1}}(y,x) \vee  \mu_{S_{-1}}(y,x)  = \mu_R(x,y) \vee \mu_S(x,y) = \mu_{R \cup S}(x,y)$$ 


$\therefore$ son iguales.
\newline

La expresi�n discreta del problema es usando la forma matricial de $R$ y $S$ que se definen como:

 $R=\left[
      \begin{array}{ccc}
        r_{1, 1} & \cdots & r_{1, m} \\
        \vdots   & \ddots & \vdots       \\
        r_{n, 1} & \cdots & r_{n, m}
      \end{array}
    \right] , S=\left[
      \begin{array}{ccc}
        s_{1, 1} & \cdots & s_{1, m} \\
        \vdots   & \ddots & \vdots       \\
        s_{n, 1} & \cdots & s_{n, m}
      \end{array}
    \right] $

Se tiene que:
\begin{itemize}
\item $(R \cup S)^{-1} =\left[
      \begin{array}{ccc}
       max(r_{1, 1}.s_{1,1}) & \cdots & max(r_{1, m},s_{1,m}) \\
        \vdots   & \ddots & \vdots       \\
        max(r_{n, 1},s_{n,1}) & \cdots & max(r_{n, m},s_{n,m})
      \end{array}
    \right] ^T$
   
  $(R \cup S)^{-1}  =\left[
      \begin{array}{ccc}
       max(r_{1, 1}.s_{1,1}) & \cdots & max(r_{n, 1},s_{n,1}) \\
        \vdots   & \ddots & \vdots       \\
        max(r_{1, m},s_{1,m})  & \cdots & max(r_{n, m},s_{n,m})
      \end{array}
    \right] $

    
\item $R^{-1} \cup S^{-1}$
\newline


Se tiene que:
 $R^{-1}=\left[
      \begin{array}{ccc}
        r_{1, 1} & \cdots & r_{n, 1} \\
        \vdots   & \ddots & \vdots       \\
        r_{1, m}  & \cdots & r_{n, m}
      \end{array}
    \right] , S^{-1}=\left[
      \begin{array}{ccc}
        s_{1, 1} & \cdots & s_{n, 1}\\
        \vdots   & \ddots & \vdots       \\
         s_{1, m}   & \cdots & s_{n, m}
      \end{array}
    \right] $
    
    
   De lo anterior se tiene,
   
 $R^{-1} \cup S^{-1} =\left[
      \begin{array}{ccc}
       max(r_{1, 1}.s_{1,1}) & \cdots & max(r_{n, 1},s_{n,1}) \\
        \vdots   & \ddots & \vdots       \\
        max(r_{1, m},s_{1,m})  & \cdots & max(r_{n, m},s_{n,m})
      \end{array}
    \right] $
    
    $\therefore$ se cumple.


\end{itemize}

\item{\color{red} $\overline{R^{-1}} = (\overline{R})^{-1}$}
\newline
Se tiene que:
$$\mu_{\overline{R^{-1}}}=1-\mu_{R^{-1}}(y,x)=1-\mu_R(x,y)$$
y por otra parte:

$$\mu_{(\overline{R})^{-1}}(y,x)=\mu_{\overline{R}}(x,y)=1-\mu_R(x,y)$$
 $\therefore$ se cumple.
 
 La expresi�n discreta del problema puede ser representada por:
 
$$\overline{R^{-1}}=U-R^{T}=\left[
      \begin{array}{ccc}
        1-r_{1, 1} & \cdots & 1-r_{n, 1} \\
        \vdots   & \ddots & \vdots       \\
        1-r_{1, m}  & \cdots & 1-r_{n, m}
      \end{array}
    \right] $$
   
   
   Donde $U=\left[
      \begin{array}{ccc}
        1 & \cdots & 1 \\
        \vdots   & \ddots & \vdots       \\
        1  & \cdots & 1
      \end{array}
    \right]$
    
  Por otra parte:
$$ (\overline{R})^{-1}=(U-R)^T= \left[
      \begin{array}{ccc}
        1- r_{1, 1} & \cdots & 1- r_{1, m} \\
        \vdots   & \ddots & \vdots       \\
        1- r_{n, 1} & \cdots & 1-r_{n, m}
      \end{array}
    \right]^T=\left[
      \begin{array}{ccc}
        1-r_{1, 1} & \cdots & 1-r_{n, 1} \\
        \vdots   & \ddots & \vdots       \\
        1-r_{1, m}  & \cdots & 1-r_{n, m}
      \end{array}
    \right] $$
  
$\therefore$ se cumple

\end{enumerate}

\section{Pregunta 4}
Sean $A$,$A'$ Fuzzy sets y sean $R$, $R'$ relaciones Fuzzy de X en Y y S,T relaciones Fuzzy de Y en Z. Verificar que:

\begin{itemize}
\item {\color{red}$(R \circ S) ^{-1} = S^{-1} \circ R^{-1} $}

Daro que la relaci�n $R \circ S$ es una relaci�n de $X$ en $Z$, la relaci�n inversa $(R \circ S) ^{-1}$ es una relaci�n que va de $Z$ en $X$.

Se define la funci�n de pertenencia de la relaci�n $(R \circ S) ^{-1}$ como:

\begin{equation}
\label{eq:41}
\mu_{(R \circ S)^{-1}}(z,x)=\mu_{R \circ S}(x,z)
\end{equation}

donde

$$\mu_{R \circ S}(x,z)=max_{y \in Y}[\mu_R(x,y) \wedge \mu_S(y,z)]$$

Por otra parte la relaci�n $S^{-1} \circ R^{-1}$ se define como:

$$\mu_{S^{-1} \circ R^{-1}} = max_{y \in Y}[\mu_{S^{-1}}(z,y) \wedge \mu_{R^{-1}}(y,x)] $$

Considerando las relaciones fuzzy converse de $S^{-1}$ y $R^{-1}$ se tiene:

$$\mu_{S^{-1} \circ R^{-1}} = max_{y \in Y}[\mu_{S}(y,z) \wedge \mu_{R}(x,y)] $$

$$\mu_{S^{-1} \circ R^{-1}} = max_{y \in Y}[\mu_{R}(x,y) \wedge \mu_{S}(y,z)] $$

Que corresponde a la equivalencia mostrada en la ecuaci�n \ref{eq:41}. $\therefore$ se cumple.


\item {\color{red}$(R \circ S) \neq S \circ R$}

La desigualdad se cumple ya que la relaci�n $R \circ S$ est� definida en $X \times Z$ y la relaci�n $S \circ R$ ni siquiera es v�lida, debido a que la relaci�n $S$ est� definida en $Y \times Z$ y la relaci�n $R$ en $X \times Y$ ($Z$ y $X$ son distintos). 

\item {\color{red}$S \subseteq T \Rightarrow R \circ S \subseteq  R \circ T$}

Se tiene que la relaci�n de inclusi�n entre $S$ y $T$ se define como:

$$S \subseteq T \Leftrightarrow \mu_S(y,z) \leq \mu_T(y,z) \quad \forall x \in X \quad \forall y \in Y$$

$$\Rightarrow min(\mu_R(x,y),\mu_S(y,z)) \leq min(\mu_R(x,y),\mu_T(y,z))$$

$$\Rightarrow max_{y \in Y}\{min(\mu_R(x,y),\mu_S(y,z))\} \leq max_{y \in Y}\{min(\mu_R(x,y),\mu_T(y,z))\}$$

$$\Rightarrow \mu_{R \circ S}(x,z) \leq \mu_{R \circ T}(x,z)$$

$$\Rightarrow R \circ S \subseteq  R \circ T$$


$\therefore$ la propiedad se cumple.

\end{itemize}

\end{document}


