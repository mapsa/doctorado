\documentclass[12pt]{amsart}

\voffset=-10mm%
\oddsidemargin=0pt
\evensidemargin=0pt
\setlength{\textwidth}{167mm}
\setlength{\textheight}{230mm}
\usepackage[spanish]{babel} %Biblioteca de espa�ol
\usepackage[latin 1]{input enc} %Biblioteca de espa�ol
\usepackage{amscd,amssymb,amsthm}

\include{epsf}
\usepackage{amsmath}
\usepackage{amsfonts}

\newtheorem{theorem}{Theorem}[section]
\newtheorem{lemma}{Lemma}[section]
\newtheorem{corollary}{Corollary}[section]
%\theoremstyle{conjecture}
\newtheorem{conjecture}{Conjecture}[section]

\theoremstyle{definition}
\newtheorem{definition}{Definition}[section]
\newtheorem{example}[theorem]{Example}
\newtheorem{xca}[theorem]{Exercise}
\theoremstyle{remark}
\newtheorem{remark}[theorem]{Remark}
%\numberwithin{equation}{section}

\DeclareMathOperator*{\Iint}{\iint}

\def\proof{\noindent{\bf Proof.\ \ }}
\def\phi{\varphi}
\def\epsilon{\varepsilon}
\def\RE{\mbox{\rm Re}\,}
\def\IM{\mbox{\rm Im}\,}
\def\AZ{|z|}
\def\COG{\CC_{\Omega{_\gamma}}}
\def\COOG{\CC(\Omega_\gamma)}
\def\CU{\CC_u}
\def\CUU{\CC_u'}
\def\CHL{\CH(\Lambda)}
\def\CHO{\CH(\Omega_\gamma^*)}
\def\HY{{}_2F_1}
%-------------------------mathcal

\def\A{{\mathcal A }}
\def\CW{{\mathcal W }}
\def\CH{{\mathcal H }}
\def\CHN{\CH'}
\def\CP{{\mathcal P }}
\def\CO{{\mathcal O }}
\def\CS{{\mathcal S }}
\def\CL{{\mathcal L }}
\def\CC{{\mathcal C }}
\def\CW{{\mathcal W}}
\def\CR{{\mathcal R}}
\def\CRu{{\mathcal R}^u}
\def\CT{{\mathcal T}}
\def\CTT{\widetilde{{\mathcal T}}}
\def\WO{{\mathcal W}_{\Omega }}


%-------------------------------Bbb
\def\C{{\mathbb C }}
\def\N{{\mathbb N }}
\def\R{{\mathbb R }}
\def\U{{\mathbb U }}
\def\D{{\mathbb D }}
%----------------------------Zusammengesetztes

\def\H{{\mathcal H}(\D)}
\def\Hyp{{}_2F_1}
\def\Hb{{\mathcal H}(\DB)}
\def\HO{{\mathcal H}_0(\D )}
\def\Hoo#1{{\mathcal H}(\overline{\D_{#1}})}
\def\HB#1{{\mathcal H}(\D_{#1})}
\def\Ho#1{{\mathcal H}({#1})}
\def\d#1{\D_{#1}}
\def\DB{\overline{\D}}
\def\Db#1{\overline{\D_{#1}}}
\def\DBR{\overline{\D_\rho}}

\def\DV{de la Vall\'ee Poussin }
\def\F{{}_2F_1}
\def\zz{\overline{z}}
\def\Li{\mbox{\rm Li}_}
\newcommand{\mvec}[1]{\mbox{\bfseries\itshape #1}}
\newcommand{\dd}{\mathop{}\!\mathrm{d}}
\newcommand{\ee}{\mathrm{e}}
\newcommand{\res}{\mathrm{res}}
\newcommand{\Ln}{\ell n\,}
 
%-------------------------------------------------------

\title{L�gicas Temporales}

\author{Paola Arce}
\address{Departamento de Inform\'atica,
         Universidad T\'ecnica F. Santa Mar\'\i a, 
         Valpara\'\i so, Chile}
\email{paola.arce@usm.cl}

\begin{document}

\begin{abstract}
We prove $1+1=2$.
\end{abstract}





\maketitle

\section{Introduction}

El t�rmino l�gica temporal (temporal logic) se utiliza para representar informaci�n de manera temporal en un contexto de l�gica. La l�gica temporal es un tipo de l�gica modal introducida en 1960 por Arthur Prior bajo el nombre de Tense Logic \cite{sep-logic-temporal} que surge del inter�s en la relaci�n entre el tiempo y la modalidad (atribuido al fil�sofo Diodorus Cronus).
En 1977 Pnueli propone el uso de l�gicas temporales para verificaci�n de propiedades a partir de la especificaci�n de un sistema a partir de axiomas. Este proceso fue optimizado por Clarke y Emerson en 1981 para ciertas l�gicas temporales. Posteriormente comenz� el estudio de la complejidad de verificaci�n de estas l�gicas en t�cnicas como: model-checking, parametrized complexity, etc.

La l�gica temporal es usada para aclarar aspectos filos�ficos acerca del tiempo, para definir expresiones sem�nticas temporales en lenguaje natural, para codificar conocimiento temporal en inteligencia artificial y como herramienta para manejar los aspectos temporales de la ejecuci�n de programas computacionales. 

\section{Sistemas de transici�n}
Las l�gicas temporales permiten especificar propiedades din�micas de un sistema sin introducir el tiempo expl�citamente, describiendo secuencias de transiciones entre estados en un sistema que evoluciona en el tiempo. Por lo tanto, para poder estudiar las transiciones entre los estados, se debe primero modelar los sistemas como sistemas de transici�n.

Un sistema de transici�n o estructura de Kripke sobre un alfabeto finito $\Sigma$ es una estructura  $M=(E,(P_a)_{a \in \Sigma},R)$, donde:

\begin{itemize}
\item $E$ es un conjunto finito de estados
\item $P_a$ es un predicado en $E$ ($P_a \subseteq E$)
\item $R$ es una relaci�n de transici�n binaria en $E$
\end{itemize}

\bibliographystyle{unsrt}
 \bibliography{templogic}

\end{document}
